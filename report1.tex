\documentclass[a4paper,12pt,titlepage,final]{article}
%% тут была опечатка                      ^
%%----------------------------------------- 

\usepackage[T1,T2A]{fontenc}     % форматы шрифтов
\usepackage[russian]{babel}      % пакет русификации
\usepackage{tikz}                % для создания иллюстраций
\usepackage{pgfplots}            % для вывода графиков функций
\usepackage{geometry}		 % для настройки размера полей
\usepackage{indentfirst}         % для отступа в первом абзаце секции
\usepackage{multirow}            % для таблицы с результатами


\usepackage{fontspec}
\setmainfont{Comic Sans MS}   %% нужен какой-нибудь установленный _в системе_ шрифт
                           %% с поддержкой русского


% выбираем размер листа А4, все поля ставим по 3см
\geometry{a4paper,left=30mm,top=30mm,bottom=30mm,right=30mm}

\setcounter{secnumdepth}{0}      % отключаем нумерацию секций


% У МЕНЯ ТАКОЙ НЕ ОКАЗАЛОСЬ
%%%\usepgfplotslibrary{fillbetween} % для изображения областей на графиках

\begin{document}
% Титульный лист
\begin{titlepage}
    \begin{center}
	{\small \sc Московский государственный университет \\имени М.~В.~Ломоносова\\
	Факультет вычислительной математики и кибернетики\\}
	\vfill
	{\Large \sc Отчет по заданию №1}\\
	~\\
	{\large \bf <<Методы сортировки>>}\\ 
	~\\
	{\large \bf Вариант 2 / 4 / 2 / 4}
    \end{center}
    \begin{flushright}
	\vfill {Выполнил:\\
	студент 103 группы\\
	Никифоров~Н.~И.\\
	~\\
	Преподаватель:\\
	Кузьменкова~Е.~А.}
    \end{flushright}
    \begin{center}
	\vfill
	{\small Москва\\2018}
    \end{center}
\end{titlepage}

% Автоматически генерируем оглавление на отдельной странице
\tableofcontents
\newpage

\section{Постановка задачи}

В данном разделе необходимо четко написать, какая задача решалась. В частности, необходимо
\begin{itemize}
\item сказать, что требуется реализовать два метода сортировки массива чисел и провести их
    экспериментальное сравнение,
\item назвать конкретные методы, которые сравниваются,
\item сказать, какого типа элементы массивов,
\item сказать, в каком порядке сортировать.
\end{itemize}
Все выше перечисленное должно быть оформлено в виде грамматически и логически
связного текста на русском языке. Копирования данного фрагмента из описания задания 
с подстановкой названия метода сортировки недостаточно.

\newpage

\section{Результаты экспериментов}

В данном разделе необходимо привести результаты экспериментов, теоретические оценки и
провести их сравнение. Теоретические оценки количества сравнений и обменов можно брать
из литературы, но в этом случае необходимо приводить ссылку не соответствующую литературу.
Ссылка выглядит так~\cite{cs}.

\begin{table}[h]
\centering
\begin{tabular}{|c|c|c|c|c|c|c|c|}
    \hline
    \multirow{2}{*}{\textbf{n}} & \multirow{2}{*}{\textbf{Параметр}} & \multicolumn{4}{|c|}{\textbf{Номер сгенерированного массива}} & \textbf{Среднее} \\
    \cline{3-6}
    & & \parbox{1.5cm}{\centering 1} & \parbox{1.5cm}{\centering 2} & \parbox{1.5cm}{\centering 3} & \parbox{1.5cm}{\centering 4} & \textbf{значение} \\
    \hline
    \multirow{2}{*}{10} & Сравнения & 65 & 65 & 65 & 65 & 65 \\
    \cline{2-7}
                        & Перемещения & 0 & 5 & 5 & 6 & 4 \\
    \hline
    \multirow{2}{*}{100} & Сравнения & 5150 & 5150 & 5150 & 5150 & 5150 \\
    \cline{2-7}
                         & Перемещения & 0 & 50 & 94 & 97 & 60 \\
    \hline
    \multirow{2}{*}{1000} & Сравнения & 501500 & 501500 & 501500 & 501500 & 501500 \\
    \cline{2-7}
                          & Перемещения & 0 & 500 & 991 & 994 & 621 \\
    \hline
    \multirow{2}{*}{10000} & Сравнения & 50015000 & 50015000 & 50015000 & 50015000 & 50015000\\
    \cline{2-7}
                           & Перемещения & 0 & 5000 & 9994 & 9990 & 6246 \\
    \hline
\end{tabular}
\caption{Результаты работы сортировки простым выбором с добавленной оптимизацией.}
\end{table}
Тестовые данные полностью соответствуют тереоритическим.

\begin{table}[h]
\centering
\begin{tabular}{|c|c|c|c|c|c|c|c|}
    \hline
    \multirow{2}{*}{\textbf{n}} & \multirow{2}{*}{\textbf{Параметр}} & \multicolumn{4}{|c|}{\textbf{Номер сгенерированного массива}} & \textbf{Среднее} \\
    \cline{3-6}
    & & \parbox{1.5cm}{\centering 1} & \parbox{1.5cm}{\centering 2} & \parbox{1.5cm}{\centering 3} & \parbox{1.5cm}{\centering 4} & \textbf{значение} \\
    \hline
    \multirow{2}{*}{10} & Сравнения & 45 & 45 & 30 & 25 & 36 \\
    \cline{2-7}
                        & Перемещения & 54 & 29 & 21 & 21 & 31 \\
    \hline
    \multirow{2}{*}{100} & Сравнения & 4950 & 4950 & 830 & 746 & 2896 \\
    \cline{2-7}
                         & Перемещения & 5049 & 2549 & 491 & 418 & 2126 \\
    \hline
    \multirow{2}{*}{1000} & Сравнения & 499500 & 499500 & 11204 & 11307 & 252500 \\
    \cline{2-7}
                          & Перемещения & 500499 & 250499 & 5734 & 5071 & 190450 \\
    \hline
    \multirow{2}{*}{10000} & Сравнения & 49995000 & 49995000 & 165070 & 150307 & 25076259 \\
    \cline{2-7}
                           & Перемещения & 50004999 & 25004999 & 99510 & 80439 & 18800486 \\
    \hline
\end{tabular}
\caption{Результаты работы быстрой сортировки.}
\end{table}


\newpage

\section{Структура программы и спецификация функций}

Программа состоит из некоторых функций:
\begin{itemize}
  \item {\bf void swap(long long *a, long long *b)} меняет два числа по данным указателям.
  \item {\bf long long mod(long long a)} возвращает модуль числа.
  \item {\bf void print\_array(long long *a, int n)} печатает данный массив длины n.
  \item {\bf long long comp(long long a, long long b)} возвращает разность модулей двух чисел.
  \item {\bf int find\_max\_index(long long *a, int start, int n)} находит индекс максимального элемента в данной части массива.
  \item {\bf void sort\_vst(long long *a, int n)} Сортировка выбором.
  \item {\bf int partition(long long *a, int p, int r)} Основная часть быстрой сортировки, которая выбирает опорный элемент в данном куске массива. И расставляет остальные относительно него.
  \item {\bf void quicksort(long long *a, int p, int r)} Быстрая сортировка.
  \item {\bf void quick\_sort(long long *a, int n)} Оболочка для быстрой сортировки.
  \item {\bf void gen\_obrt(long long *a, long long *b, int n)} Генерация массива с обратным порядком элементов.
  \item {\bf void gen\_norm(long long *a, long long *b, int n)} Генерация массива с прямым порядком элементов.
  \item {\bf void gen\_rand(long long *a, long long *b, int n)} Генерация массива со случайными элементами.
\end{itemize}

\newpage

\section{Отладка программы, тестирование функций}

Сначала была написаана сортировка выбором. Потом были написаны три функции генерации массивов.
На трёх вариантах массива была протестировна первая сортировка вставками. Правильность сортировки была проверена специальной функцией, которая ныне удалена. Затем была написана быстрая сортировка. Которая также тестировалась на трёх вариантах массива. 

\newpage

\section{Анализ допущенных ошибок}


\newpage
\begin{raggedright}
\addcontentsline{toc}{section}{Список цитируемой литературы}
\begin{thebibliography}{99}
\bibitem{cs} Кормен Т., Лейзерсон Ч., Ривест Р, Штайн К. Алгоритмы: построение и анализ.
    Второе издание.~--- М.:<<Вильямс>>, 2005.
\end{thebibliography}
\end{raggedright}

\newpage

\section*{Требования к оформлению}

В данном разделе приводятся общие требования к оформлению текста отчета.
Данный раздел не должен включаться в сдаваемый отчет.

\begin{enumerate}
\item Отчет оформляется на листах A4. Поля должны составлять от 2 до 4
    сантиметров и быть одинаковыми на всех страницах отчета.
\item Основной текст отчета оформляется пропорциональным шрифтом с засечками,
    таким как Times New Roman. Размер шрифта может составлять либо 12pt, либо 14pt.
    Межстрочные интервалы могут быть единичными или полуторными в случае 12-го шрифта
    и только единичными в случае использования 14-го шрифта.
\item Никаких дополнительных межстрочных интервалов между абзацами не делается.
    Первая строка абзаца должна иметь небольшой отступ (5-10мм), одинаковый для
    всех абзацев, включая первый абзац раздела.
\item Заголовки первого уровня должны быть набраны более крупным шрифтом (16pt или 18pt).
    В заголовках допускается использование как основного шрифта, так и пропорционального
    шрифта без засечек, такого как Arial. Все заголовки всех уровней должны быть набраны
    одним шрифтом. Размер шрифта заголовков большего уровня не должен превосходить размер
    шрифта заголовков меньшего уровня.
\item Фрагменты программ и сценариев сборки должны быть набраны моноширинным шрифтом, таким
    как Courier. Размер шрифта, используемый в листингах программ может отличаться от размера,
    использованного при наборе основного текста, но должен быть одинаковым во всех частях
    отчета и принадлежать интервалу от 10pt до 14pt.
\item Выделение полужирным и/или курсивом допускается для отдельных слов в основном тексте,
    если это требуется. Заголовки рекомендуется выделять жирным.
\item Основной текст выравнивается по двум сторонам (по ширине). На титульном листе часть текста
    выравнивается по центру, часть по правом краю. Список литературы и названия разделов 
    выравниваются по левому краю.
\item Таблицы и рисунки выравниваются по центру. Все таблицы и рисунки должны быть пронумерованы
    и подписаны. Нумерация сквозная, отдельная для рисунков и таблиц, арабскими цифрами.
\item При использовании растровых изображения для иллюстраций в отчете
    необходимо обеспечить достаточное разрешение этих изображений. Качество изображения
    считается достаточным, если все надписи на нем легко читаются. Если на тексте, содержащемся
    на рисунке, явно заметно размазывание элементов букв, то такое изображение считается
    слишком низкого качества, и оно не должно быть использовано в отчете.
\item Таблицы должны быть сверстаны как таблицы, а не вставлены как рисунки.
\item Список литературы должен содержать для книг и статей (в соответствующем порядке).
    \begin{itemize}
    \item Фамилии и инициалы (либо полные имена) всех авторов.
    \item Название книги или статьи.
    \item Название журнала и номер тома или выпуска для статей.
    \item Город и год издания.
    \end{itemize}
\item Список литературы для электронных источников должен содержать
    \begin{itemize}
    \item Название страницы.
    \item Полный адрес страницы.
    \item Дата обращения.
    \end{itemize}
\item Ссылки на Википедию и другие электронные ресурсы для оценок численных 
    методов не принимаются. Используйте книги и/или научные статьи в качестве 
    источников данной информации.
\item На все элементы списка литературы должны присутствовать ссылки в тексте отчета. Элементы
    списка литературы должны идти в том порядке, в котором ссылки на них первый
    раз встречаются в тексте.
\item Титульный лист оформляется следующим образом.
    \begin{itemize}
    \item Сверху с выравниванием по центру пишется название ВУЗа и факультета. Данный
        фрагмент пишется заглавными или малыми заглавными буквами.
    \item В центре страницы располагается следующая информация (сверху вниз).
        \begin{itemize}
        \item Наименование работы (<<Отчет по заданию №1>>, без кавычек заглавными
            или малыми заглавными буквами).
        \item Тема работы (<<Методы сортировки>>, в кавычках, жирным шрифтом).
        \item Вариант. (Без кавычек жирным шрифтом).
        \end{itemize}
    \item Информация о студенте, выполнившем работу и преподавателе выравнивается
        по правому краю. Данный фрагмент набирается обычным шрифтом.
    \item Внизу страницы с выравниванием по центру обычным или немного уменьшенным
        шрифтом пишется город и год выполнения работы.
    \end{itemize}
\end{enumerate}


\end{document}

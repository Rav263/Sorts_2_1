\documentclass[a4paper,12pt,titlepage,final]{article}
%% тут была опечатка                      ^
%%----------------------------------------- 

\usepackage[T1,T2A]{fontenc}     % форматы шрифтов
\usepackage[russian]{babel}      % пакет русификации
\usepackage{tikz}                % для создания иллюстраций
\usepackage{pgfplots}            % для вывода графиков функций
\usepackage{geometry}		 % для настройки размера полей
\usepackage{indentfirst}         % для отступа в первом абзаце секции
\usepackage{multirow}            % для таблицы с результатами


\usepackage{fontspec}
\setmainfont{Times New Roman}   %% нужен какой-нибудь установленный _в системе_ шрифт
\setmonofont{Courier}%% с поддержкой русского


% выбираем размер листа А4, все поля ставим по 3см
\geometry{a4paper,left=30mm,top=30mm,bottom=30mm,right=30mm}

\setcounter{secnumdepth}{0}      % отключаем нумерацию секций


% У МЕНЯ ТАКОЙ НЕ ОКАЗАЛОСЬ
%%%\usepgfplotslibrary{fillbetween} % для изображения областей на графиках

\begin{document}
% Титульный лист
\begin{titlepage}
    \begin{center}
	{\small \sc Московский государственный университет \\имени М.~В.~Ломоносова\\
	Факультет вычислительной математики и кибернетики\\}
	\vfill
	{\Large \sc Отчет по заданию №1}\\
	~\\
	{\large \bf <<Методы сортировки>>}\\ 
	~\\
	{\large \bf Вариант 2 / 4 / 2 / 4}
    \end{center}
    \begin{flushright}
	\vfill {Выполнил:\\
	студент 103 группы\\
	Никифоров~Н.~И.\\
	~\\
	Преподаватель:\\
	Кузьменкова~Е.~А.}
    \end{flushright}
    \begin{center}
	\vfill
	{\small Москва\\2018}
    \end{center}
\end{titlepage}

% Автоматически генерируем оглавление на отдельной странице
\tableofcontents
\newpage

\section{Постановка задачи}

В данной работе решалась задача сравнения двух сортировок массивов чисел. Было дано два метода сортировки:
\begin{itemize}
  \item Сортировка простым выбором.
  \item Метод быстрой сортировки.
\end{itemize}
Для начала нужно было реализовать эти два метода и проверить, что они работают верно при помощи эксперимента.
Эксперимент проводился на трёх вариантах массивов.
\begin{itemize}
  \item Массив с прямым порядком элементов.
  \item Массив с обратным порядком элементов.
  \item Массмв со случайными элементами.
\end{itemize}
Все элементы каждого массива имели тип {\bf \ttfamily long long int}.
Исходный массив требовалось отсортировать в порядке не возростания модулей чисел.


\newpage

\section{Результаты экспериментов}

В данном разделе приведены теоретические оценки и практические данные, так же выполнено их сравнение.

\begin{table}[h]
\centering
\begin{tabular}{|c|c|c|c|c|c|c|c|}
    \hline
    \multirow{2}{*}{\textbf{n}} & \multirow{2}{*}{\textbf{Параметр}} & \multicolumn{4}{|c|}{\textbf{Номер сгенерированного массива}} & \textbf{Среднее} \\
    \cline{3-6}
    & & \parbox{1.5cm}{\centering 1} & \parbox{1.5cm}{\centering 2} & \parbox{1.5cm}{\centering 3} & \parbox{1.5cm}{\centering 4} & \textbf{значение} \\
    \hline
    \multirow{2}{*}{10} & Сравнения & 65 & 65 & 65 & 65 & 65 \\
    \cline{2-7}
                        & Перемещения & 0 & 5 & 5 & 6 & 4 \\
    \hline
    \multirow{2}{*}{100} & Сравнения & 5150 & 5150 & 5150 & 5150 & 5150 \\
    \cline{2-7}
                         & Перемещения & 0 & 50 & 94 & 97 & 60 \\
    \hline
    \multirow{2}{*}{1000} & Сравнения & 501500 & 501500 & 501500 & 501500 & 501500 \\
    \cline{2-7}
                          & Перемещения & 0 & 500 & 991 & 994 & 621 \\
    \hline
    \multirow{2}{*}{10000} & Сравнения & 50015000 & 50015000 & 50015000 & 50015000 & 50015000\\
    \cline{2-7}
                           & Перемещения & 0 & 5000 & 9994 & 9990 & 6246 \\
    \hline
\end{tabular}
\caption{Результаты работы сортировки простым выбором.}
\end{table}
Теоретическая оценка количества сравнений O($N^2$) Теоретическая оценка количества обменов O($N$) ~\cite{cs}. Данные, полученные экспериментальным путем, соответствуют теоретическим.

\begin{table}[h]
\centering
\begin{tabular}{|c|c|c|c|c|c|c|c|}
    \hline
    \multirow{2}{*}{\textbf{n}} & \multirow{2}{*}{\textbf{Параметр}} & \multicolumn{4}{|c|}{\textbf{Номер сгенерированного массива}} & \textbf{Среднее} \\
    \cline{3-6}
    & & \parbox{1.5cm}{\centering 1} & \parbox{1.5cm}{\centering 2} & \parbox{1.5cm}{\centering 3} & \parbox{1.5cm}{\centering 4} & \textbf{значение} \\
    \hline
    \multirow{2}{*}{10} & Сравнения & 45 & 45 & 30 & 25 & 36 \\
    \cline{2-7}
                        & Перемещения & 54 & 29 & 21 & 21 & 31 \\
    \hline
    \multirow{2}{*}{100} & Сравнения & 4950 & 4950 & 830 & 746 & 2896 \\
    \cline{2-7}
                         & Перемещения & 5049 & 2549 & 491 & 418 & 2126 \\
    \hline
    \multirow{2}{*}{1000} & Сравнения & 499500 & 499500 & 11204 & 11307 & 252500 \\
    \cline{2-7}
                          & Перемещения & 500499 & 250499 & 5734 & 5071 & 190450 \\
    \hline
    \multirow{2}{*}{10000} & Сравнения & 49995000 & 49995000 & 165070 & 150307 & 25076259 \\
    \cline{2-7}
                           & Перемещения & 50004999 & 25004999 & 99510 & 80439 & 18800486 \\
    \hline
\end{tabular}
\caption{Результаты работы быстрой сортировки.}
\end{table}
Теоритическая оценка количества сравнение в среднем O($N*log(N)$). Теоретическая оценка количества обменов O($N*log(N)/6$). В худшем случае оценка O($N^2$) ~\cite{as}. Данные, полученные экспериментальным путём, соответствуют теоретическим.

\newpage

\section{Структура программы и спецификация функций}

Программа состоит из некоторых функций:
\begin{itemize}
  \item {\bf \ttfamily void swap(long long *a, long long *b)} меняет два числа по данным указателям.
  \item {\bf \ttfamily long long mod(long long a)} возвращает модуль числа.
  \item {\bf \ttfamily void print\_array(long long *a, int n)} печатает данный массив длины n.
  \item {\bf \ttfamily long long comp(long long a, long long b)} возвращает разность модулей двух чисел.
  \item {\bf \ttfamily int find\_max\_index(long long *a, int start, int n)} находит индекс максимального элемента в данной части массива.
  \item {\bf \ttfamily void sort\_vst(long long *a, int n)} Сортировка выбором.
  \item {\bf \ttfamily int partition(long long *a, int p, int r)} Основная часть быстрой сортировки, которая выбирает опорный элемент в данном куске массива. И расставляет остальные относительно него.
  \item {\bf \ttfamily void quicksort(long long *a, int p, int r)} Быстрая сортировка.
  \item {\bf \ttfamily void quick\_sort(long long *a, int n)} Оболочка для быстрой сортировки.
  \item {\bf \ttfamily void gen\_obrt(long long *a, long long *b, int n)} Генерация массива с обратным порядком элементов.
  \item {\bf \ttfamily void gen\_norm(long long *a, long long *b, int n)} Генерация массива с прямым порядком элементов.
  \item {\bf \ttfamily void gen\_rand(long long *a, long long *b, int n)} Генерация массива со случайными элементами.
\end{itemize}

\newpage

\section{Отладка программы, тестирование функций}

Разработка программы велась в несколько этапов:
\begin{itemize}
  \item Написание первого типа сортировки (Сортировка выбором)
  \item Тестирование первого типа сортировки на случайных массивах разной длины.
  \item Написание второго типа сортировки (Быстрая сортировка)
  \item Тестирование второго типа сортировки на случайных массивах разной длины.
  \item Написание функций генерирования разных типов массивов.
  \item Тестирование сортировок на этих массивах и запись эксперементальных данных.
  \item Написание отчёта о работе.
\end{itemize}
Приведём результаты тестирования для всех типов массивов длины 10:
\begin{itemize}
  \item Тип 1, исходный массив:      10 9 8 7 6 5 4 3 2 1
  \item Тип 1, результат сортировки: 10 9 8 7 6 5 4 3 2 1
  \item Тип 2, исходный массив:      1 2 3 4 5 6 7 8 9 10
  \item Тип 2, результат сортировки: 10 9 8 7 6 5 4 3 2 1
  \item Тип 3, исходный массив (случайные данные взяты по модулю 100): -87 -83 -66 17 40 -69 2 3 16 -69 
  \item Тип 3, результат сортировки: -87 -83 -69 -69 -66 40 17 16 3 2
\end{itemize}

\newpage

\section{Анализ допущенных ошибок}
Исправленны ошибки в отчёте после комментария преподавателя.
\begin{itemize}
  \item Добавлены данные тестирования сортировок.
\end{itemize}

\newpage
\begin{raggedright}
\addcontentsline{toc}{section}{Список цитируемой литературы}
\begin{thebibliography}{99}
\bibitem{cs} Кормен Т., Лейзерсон Ч., Ривест Р, Штайн К. Алгоритмы: построение и анализ.
    Второе издание.~--- М.:<<Вильямс>>, 2005.
\bibitem{as} Вирт Н. Алгоритмы и структуры данных. — М.: Мир, 1989.
\end{thebibliography}
\end{raggedright}

\end{document}
